\section{Schlussbemerkung}
\label{sec:Schlussbemerkung}

Die Untersuchung der berührungslosen Energieübertragung hat ergeben, dass eine Simulation als gute Annäherung an die Realität dient. Je näher man mit der Simulation an die Realität gelangen will, desto mehr Faktoren müssen berücksichtigt werden und dementsprechend aufwendig wird das Simulationsmodell. In der Praxis kommt es oft zu geringen Abweichungen mit unterschiedlich starken Auswirkungen. Es hat sich herausgestellt, dass dem Kopplungsfaktor bei den untersuchten Schaltungstypen ein entscheidender Faktor zukommt. Ist dieser zu tief, entstehen hohe Streuinduktivitäten. Diese führen zu hohen Verlusten auf der Primärseite. Aus diesem Grund eignet sich der Flyback-Converter für diese Anwendung nicht.
\newline \\
Bei der berührungslosen Datenübertragung konnte die geforderte Geschwindigkeit von \linebreak \SI{31.25}{MHz} erreicht und übertroffen werden. Mit der Testschaltung konnte ein Rechtecksignal von \SI{45}{MHz} über eine Distanz von \SI{4}{cm} übertragen werden. Auch eine Übertragung durch die Acryl-Scheibe ist erfolgreich getestet worden. Durch die Messungen hat sich herausgestellt, dass Umgebungslicht keinen störenden Einfluss auf die Schaltung hat.
\newline \\
Die verschiedenen Versuchsaufbauten haben gezeigt, dass die zu überwindende Distanz und somit die konstruktiven Voraussetzungen an der Maschine entscheidend für die Implementierung sind. Es ist deshalb sehr wichtig, zum Start der Bachelor-Thesis alle nötigen Abklärungen hinsichtlich der Konstruktion zu tätigen.
\newline \\
Für die Weiterführung des Projekts ist ein anderer Schaltungstyp für die Energieübertragung zu wählen. Sicherlich muss der Einsatz eines Resonanzwandlers genaustens geprüft werden. Die Kompensation der Streuinduktivität mit einem Schwingkreis macht diese Schaltung sehr interessant für diese Anwendung. Die Erfahrungen aus diesem Projekt mit den verschiedenen Simulationen und Dimensionierungen werden für die Fortsetzung der Arbeit sehr hilfreich sein.
\newline \\
Das Konzept für die berührungslose Datenübertragung kann grundsätzlich weiterverfolgt werden. Die beiden Grundschaltungen von Sender- und Empfänger können verwendet werden und auf die Anforderungen abgestimmt werden. Auch die Option einer Änderung des Bussystems vom VARAN-Bus auf den CAN-Bus wird in Betracht gezogen. Die tiefere Frequenz und die einfachere Implementierung eines Feedback-Signals für die Lastregelung sind Vorteile des CAN-Bus. Das Prinzip zur Lichtstreuung mit der Acryl-Scheibe muss noch weiter optimiert und untersucht werden. Zusammen mit dem Hersteller dürfte eine Lösung gefunden werden.
