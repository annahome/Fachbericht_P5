\section{Grundlagen}
In diesem Kapitel werden die wichtigsten Grundlagen erklärt, welche nötig sind um den Bericht zu verstehen.
\subsection{Grundlagen zur Energieübertragung}
In diesem Unterkapitel werden die Grundlagen zur Energieübertragung erläutert. Im wesentlichen beinhaltet dies folgende Themen:

\paragraph{Kopplungsfaktor}
Der Kopplungsfaktor $ k $ lässt sich wie folgt definieren:
\begin{equation}
k=\frac{M}{\sqrt{L_{1}\cdot L_{2}}}
\label{eq:kopplungsfaktor}
\end{equation}

\paragraph{Auswahl der Schaltungstopologie}
\todo[inline]{Vor und Nachteile Flayback}

\paragraph{Flyback}
\todo[inline]{Abbildung Flyback}
\todo[inline]{Formeln zur Berechnung}
\todo[inline]{Kontinuierlicher, Diskontinuierlicher Modus}
\todo[inline]{Thema Snubber}

\paragraph{Transformator Ersatzschaltbild}
\todo[inline]{Abbildung Ersatzschaltbild}
\todo[inline]{Aufzeigen der Streuinduktivität}

\subsection{Grundlagen zur Datenübertragung}
In diesem Unterkapitel werden die Grundlagen zur Datenübertragung erläutert. Im wesentlichen beinhaltet dies folgende Themen:
\paragraph{VARAN-Bus}
\paragraph{Ethernet}
\paragraph{Photodioden-Verstärker}