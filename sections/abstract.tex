\begin{abstract}
Das Institut für Automation entwickelt einen neuen Dosenverschliesser in Zusammenarbeit mit der Ferrum AG. Weil eine Schleppkette zur Kabelführung für industrielle Maschinen weniger geeignet ist, soll diese ersetzt werden. Ziel dieser Arbeit ist es, ein Konzept zur berührungslosen Daten- und Energieübertragung zu entwickeln, welches die Schleppkette ablösen kann. Dafür ist eine Leistung von mindestens \SI{300}{W}/\SI{48}{V} und eine Geschwindigkeit von \SI{31.25}{MHz} notwendig. In dieser Arbeit werden durch Simulationen und Versuchsaufbauten wichtige Erkenntnisse für die spätere Umsetzung gesammelt. Die Energieübertragung wird durch ein induktives Verfahren realisiert. Die Datenübertragung findet über den optischen Weg statt.
\newline
Die Schaltungstopologie des Flybacks zur Energieübertragung eignet sich nicht für diese Anwendung, weil durch den geringen Kopplungsfaktor hohe Verluste verursacht werden. Die Testschaltung zur Datenübertragung kann ein Rechtecksignal mit \SI{45}{MHz} übertragen und übertrifft somit die geforderte Geschwindigkeit. Acrylglas eignet sich als lichtleitendes Material und kann somit als Kanal verwendet werden.
\\
\paragraph{Keywords:} Acrylglas, Ethernet, Flyback, induktive Energieübertragung, Infrarot, Kopplungsfaktor, optische Datenübertragung, Photodiode, Transimpedanzverstärker, VARAN-Bus, 
\end{abstract}

